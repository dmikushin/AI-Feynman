%% Template for ENG 401 reports
%% by Robin Turner
%% Adapted from the IEEE peer review template

\documentclass[peerreview]{IEEEtran}
\usepackage{cite} % Tidies up citation numbers.
\usepackage{url} % Provides better formatting of URLs.
\usepackage[utf8]{inputenc} % Allows Turkish characters.
\usepackage{booktabs} % Allows the use of \toprule, \midrule and \bottomrule in tables for horizontal lines
\usepackage{graphicx}

\begin{document}

\setcounter{page}{1}

\title{Performance Optimization of the AI Feynman Symbolic Regression code}

\author{Dmitry Mikushin \\
Applied Parallel Computing LLC \\
dmitry@parallel-computing.pro \\
}
\date{\today}

\maketitle
\tableofcontents

\begin{abstract}
This report presents the AI Feynman programming code performance optimization.
\end{abstract}

\section{Overview}

AI Feynman \cite{aifeynman} is a neural network package for reconstructing original numerical expression (formula) from the resulting dataset.

\section{Program Design}

The program code is organized into the preprocessing (\emph{feature extraction}) and neural network \emph{training} phases.

The \emph{feature extraction} is given as a massive bruteforce evaluation of arbitrary basis function combinations. In order to maintain acceptable running times, this phase is implemented as a native code (Fortran). Moreover, each instance of bruteforce evaluation is limited by 30 seconds.

\section{Optimization}

AAA

\section{Preliminary results}

AAA

\begin{thebibliography}{1}
\bibitem{aifeynman}Udrescu, S.M., and Tegmark, M. 2020. AI Feynman: A physics-inspired method for symbolic regression. Science Advances, 6(16), p.eaay2631.
\end{thebibliography}

\end{document}

